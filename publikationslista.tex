\documentclass[11pt,a4paper]{article}

\usepackage{lmodern}
\usepackage{microtype}

\usepackage[a4paper,margin=1in]{geometry}

\usepackage{parskip}

\usepackage{hyperref}
\urlstyle{same}
\hypersetup{
    hidelinks,
    colorlinks=false,
}

\usepackage{sectsty}
\sectionfont{\large}


\begin{document}

\begin{center}
    \Large\bfseries
    Publikationslista -- Hans Chen
\end{center}

\section{Viktigaste bidrag till forskningen och forskarsamhället}

Min forskning stärker grunden för att förstå och förutsäga klimatvariabilitet och klimatförändring
genom att pröva etablerade hypoteser och utveckla nya analysverktyg.
Ett centralt bidrag är att jag med observationer och modeller har kvantifierat kopplingen mellan Arktis förstärkning och vädermönster på mellanbreddgraderna,
och visat att sambandet är svagt och icke-linjärt.
Denna forskning har utmanat den rådande förståelsen och bidragit till en mer nyanserad debatt.

Jag har också utvecklat nya dataassimileringsmetoder för att uppskatta koldioxidflöden mellan jordytan och atmosfären.
Till skillnad från traditionella metoder kan dessa assimilera stora datamängder från satelliter,
vilket öppnar nya möjligheter att studera kolcykeln
och övervaka mänskliga koldioxidutsläpp för att stödja Parisavtalet.

Som forskargruppsledare driver jag tvärvetenskaplig klimatforskning där vi integrerar olika perspektiv och metoder.
För mig är öppen vetenskap en kärnprincip;
jag bidrar till forskarsamhället genom att aktivt dela data och verktyg som vi utvecklar via öppna plattformar som Zenodo och GitHub.


\section{Publikationer}

\begin{enumerate}
    \item Holmgren, E., and H. W. Chen, 2025:
        A climatology of atmospheric rivers over Scandinavia and associated precipitation.
        \textit{Weather and Climate Dynamics}, \textbf{6}, 1831–1856, \url{https://doi.org/10.5194/wcd-6-1831-2025}.%
        \textsuperscript{*,**}

        {\itshape
            Jag handledde projektet och är korresponderande författare.
            Jag bidrog aktivt till utveckling av metodik,
            tolkning av resultat
            samt författande och revidering av manuskriptet.
        }

    \item Zhong, Z., H. W. Chen, A. Dai, T. Zhou, B. He, and B. Su, 2025:
        Sub-diurnal asymmetric warming has amplified atmospheric dryness since the 1980s.
        \textit{Nature Communications}, \textbf{16}, 8247, \url{https://doi.org/10.1038/s41467-025-63672-z}.%
        \textsuperscript{*,**}

        {\itshape
            Jag handledde projektet och är korresponderande författare.
            Jag bidrog aktivt till utveckling av metodik,
            tolkning av resultat
            samt författande och revidering av manuskriptet.
        }


    \item Chen, H. W., F. Zhang, T. Lauvaux, M. Scholze, K. J. Davis, and R. B. Alley, 2023:
        Regional CO\textsubscript{2} inversion through ensemble-based simultaneous state and parameter estimation: TRACE framework and controlled experiments.
        \textit{Journal of Advances in Modeling Earth Systems}, \textbf{15}, e2022MS003208, \url{https://doi.org/10.1029/2022MS003208}.

        {\itshape
            Jag konceptualiserade studien,
            utvecklade modelleringssystemet,
            utförde modellexperimenten,
            genomförde samtliga analyser,
            tolkade resultaten och skrev manuskriptet.
            Jag är korresponderande författare.
        }

    \item Fang, M., X. Li, H. W. Chen, and D. Chen, 2022:
        Arctic amplification modulated by Atlantic Multidecadal Oscillation and greenhouse forcing on multidecadal to century scales.
        \textit{Nature Communications}, \textbf{13}, 1865, \url{https://doi.org/10.1038/s41467-022-29523-x}.%
        \textsuperscript{*,**}

        {\itshape
            Jag är korresponderande författare och skrev huvuddelen av den slutgiltiga texten.
            Jag bidrog till analysen, verifiering och tolkning av resultaten.
        }

    \newpage

    \item Chen, H. W., F. Zhang, and R. B. Alley, 2016:
        The robustness of midlatitude weather pattern changes due to Arctic sea ice loss.
        \textit{Journal of Climate}, \textbf{26}, 7831--7849, \url{https://doi.org/10.1175/JCLI-D-16-0167.1}.%
        \textsuperscript{**}

        {\itshape
            Jag utformade studien tillsammans med F.Z. och R.B.A.,
            utvecklade metodiken,
            utförde modellexperimenten,
            genomförde samtliga analyser,
            tolkade resultaten med återkoppling från F.Z. och R.B.A.
            samt skrev manuskriptet.
            Jag är korresponderande författare.
        }
\end{enumerate}


\section{ORCID}

\url{https://orcid.org/0000-0002-8601-6024}


\end{document}
