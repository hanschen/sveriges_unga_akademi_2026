\documentclass[11pt,a4paper]{article}

\usepackage{lmodern}
\usepackage{microtype}

\usepackage[a4paper,margin=1in]{geometry}

\usepackage{parskip}

\usepackage{hyperref}
\urlstyle{same}
\hypersetup{
    hidelinks,
    colorlinks=false,
}

\usepackage{sectsty}
\sectionfont{\large}


\begin{document}

\begin{center}
    \LARGE\bfseries
    Publikationslista -- Hans Chen
\end{center}

\section{Viktigaste bidrag till forskningen och forskarsamhället}

Min forskning fokuserar på dynamiska processer i klimatsystemet.
Ett viktigt bidrag har varit att omvärdera kopplingen mellan Arktis uppvärmning och vädermönster på mellanlatituderna,
där vi visat att sambanden styrs av komplexa, icke-linjära processer.
Fältet lägger nu större fokus på regionala variationer och samverkande mekanismer.

Ett annat nyckelbidrag har varit att anpassa och vidareutveckla metoder från numerisk väder\-prognosmodellering
för att beräkna koldioxidflöden mellan mark och atmosfär.
Dessa metoder tar itu med en stor utmaning inom fältet --
att hantera osäkerheter i atmosfärisk transport --
och möjliggör bättre utnyttjande av satellitdata.
Operationella center som ECMWF utvecklar nu liknande metoder.

I min forskargrupp leder jag tvärvetenskaplig klimatforskning som integrerar olika perspektiv och metoder.
Våra upptäckter inkluderar en accelererad uppvärmning under dagtid jämfört med nattetid,
ångtrycksunderskottets påverkan på vegetationsproduktivitet
samt atmosfäriska floders betydelse för nederbörden i Skandinavien.


\section{Publikationer}

\begin{enumerate}
    \item Chen, H. W., F. Zhang, T. Lauvaux, M. Scholze, K. J. Davis, and R. B. Alley, 2023:
        Regional CO\textsubscript{2} inversion through ensemble-based simultaneous state and parameter estimation: TRACE framework and controlled experiments.
        \textit{Journal of Advances in Modeling Earth Systems}, \textbf{15}, e2022MS003208, \url{https://doi.org/10.1029/2022MS003208}.

        {\itshape
            Den här artikeln beskriver grunden för mitt koldioxid-dataassimileringssystem.
            Jag utformade studien,
            utvecklade metodiken med återkoppling från F.Z. och T.L.,
            utförde modellexperimenten,
            genomförde samtliga analyser,
            tolkade resultaten
            och skrev manuskriptet.
        }

    \item **Fang, M., X. Li, H. W. Chen, and D. Chen, 2022:
        Arctic amplification modulated by Atlantic Multidecadal Oscillation and greenhouse forcing on multidecadal to century scales.
        \textit{Nature Communications}, \textbf{13}, 1865, \url{https://doi.org/10.1038/s41467-022-29523-x}.

        {\itshape
            Jag bidrog till analysen och tolkningen av resultaten,
            skrev huvuddelen av det slutgiltiga manuskriptet baserat på text från M.F. och agerar som korresponderande författare.
        }

    \item **Cohen, J., X. Zhang, J. Francis, T. Jung, R. Kwok, J. Overland, T. J. Ballinger, U. S. Bhatt, H. W. Chen,
        D. Coumou, S. Feldstein, H. Gu, D. Handorf, G. Henderson, M. Ionita, M. Kretschmer, F. Laliberte, S. Lee,
        H. W. Linderholm, W. Maslowski, Y. Peings, K. Pfeiffer, I. Rigor, T. Semmler, J. Stroeve, P. C. Taylor,
        S. Vavrus, T. Vihma, S. Wang, M. Wendisch, Y. Wu, and J. Yoon, 2020:
        Divergent consensus on Arctic amplification influence on midlatitude severe winter weather.
        \textit{Nature Climate Change}, \textbf{10}, 20--29, \url{https://doi.org/10.1038/s41558-019-0662-y}.

        {\itshape
            Detta är ett exempel på samarbete inom det bredare forskningsfältet.
            Publikationen är ett resultat av en workshop som anordnades av US CLIVAR.
            Jag deltog aktivt i workshopen med särskilt fokus på klimatmodellering och robustheten i den atmosfäriska responsen på förlust av arktisk havsis.
            Efter workshopen bidrog jag med granskning och kommentarer.
        }

    \newpage

    \item Chen, H. W., F. Zhang, T. Lauvaux, K. J. Davis, S. Feng, M. P. Butler, and R. B. Alley, 2019:
        Characterization of regional-scale CO\textsubscript{2} transport uncertainties in an ensemble with flow-dependent transport errors.
        \textit{Geophysical Research Letters}, \textbf{46}, 4049--4058, \url{https://doi.org/10.1029/2018GL081341}.

        {\itshape
            Jag utformade studien,
            utvecklade metodiken med återkoppling från F.Z.,
            utförde modellexperimenten,
            genomförde samtliga analyser,
            tolkade resultaten och skrev manuskriptet.
        }

    \item Chen, H. W., F. Zhang, and R. B. Alley, 2016:
        The robustness of midlatitude weather pattern changes due to Arctic sea ice loss.
        \textit{Journal of Climate}, \textbf{26}, 7831--7849, \url{https://doi.org/10.1175/JCLI-D-16-0167.1}.

        {\itshape
            Jag utformade studien tillsammans med F.Z. och R.B.A.,
            utvecklade metodiken,
            utförde modellexperimenten,
            genomförde samtliga analyser,
            tolkade resultaten med återkoppling från F.Z. och R.B.A.
            samt skrev manuskriptet.
        }
\end{enumerate}


\section{ORCID}

\url{https://orcid.org/0000-0002-8601-6024}


\end{document}
