\documentclass[11pt,a4paper]{article}

\usepackage{lmodern}
\usepackage{microtype}
\usepackage{parskip}

\usepackage[a4paper,margin=1in]{geometry}

\usepackage{sectsty}
\sectionfont{\large}

\begin{document}

\begin{center}
    \Large\bfseries
    Ansökan -- Hans Chen
\end{center}

\section{Motivering}

\textit{Motivera kort varför du vill bli ledamot i Sveriges unga akademi. (1000 tecken)}

Jag vill bli ledamot i Sveriges unga akademi för att bidra till att stärka vetenskapens roll i en tid då kritiskt tänkande utmanas.
Som klimatforskare har jag upplevt hur vetenskap kan polariseras och angripas.
Jag disputerade i USA,
ett land med stark forskningstradition,
och det har bekymrat mig djupt att bevittna de senaste årens systematiska angrepp på vetenskap och akademisk frihet där.
Detta visar hur sårbar vetenskapen är,
och hur avgörande det är att bygga tillit till forskningsprocessen i samhället.

Jag brinner även för att förbättra villkoren för unga forskare.
Akademiska karriärvägar måste bli mer transparenta och rättvisa.
Mina erfarenheter från USA, Sverige och Kina ger mig ett brett perspektiv på olika forskningspolitiska system och möjligheter.

Genom SUA vill jag bidra till tvärvetenskapliga initiativ som stärker förtroendet för forskning
och främjar kritiskt tänkande i samhället.
I en tid av desinformation är detta avgörande för demokrati och en hållbar framtid.



\section{Aktiviteter/Frågor - bidrag}

\textit{Ge exempel på aktiviteter och frågor där du tror att du kan bidra om du blir ledamot i Sveriges unga akademi. (1000 tecken)}

1. Karriärvägar för unga forskare

SUAs bok "A Beginner’s Guide to Swedish Academia" är ett viktigt initiativ för att hjälpa unga forskare navigera det akademiska systemet.
Jag vill vidareutveckla detta arbete,
exempelvis genom att göra sådan kunskap tillgänglig via fler kanaler.


\section{Aktiviteter/Frågor – driva}

\textit{Ge exempel på aktiviteter och frågor du skulle vilja driva om du blir ledamot i Sveriges unga akademi. (1000 tecken)}

1. AI

Den snabba utvecklingen av AI omformar forskningslandskapet.
Dessa verktyg erbjuder många positiva möjligheter,
men jag är orolig för att missbruk - såsom plagiat och automatiserade granskningsprocesser - kan underminera den vetenskapliga tilliten.
Jag vill bidra till att utveckla riktlinjer och praxis som både tar vara på AIs potential och värnar forskningens integritet.

1. Förbättra etableringsmöjligheter för unga forskare

Strukturella hinder försvårar unga forskares etablering i Sverige.
Ett exempel är att vissa lärosäten tvekar att anställa forskare längre än 4 år (3+1) p.g.a. LAS,
vilket ofta krockar med kraven för många etableringsbidrag.
Jag vill arbeta både internt och i bredare akademiska sammanhang för att skapa tydligare,
hållbara och långsiktiga karriärvägar för unga forskare.

2. Attrahera och behålla toppforskare i Sverige

Ändringarna i utlänningslagen 2021 skapade stor osäkerhet för internationella doktorander när spelreglerna plötsligt ändrades mitt under deras utbildning.
Detta skadade både individer och Sveriges rykte som forskningsnation.
Regeringens utredning om förbättrade migrationsrättsliga regler är ett steg i rätt riktning,
men vi behöver skydd mot framtida regelchocker.
Jag vill arbeta för stabila och förutsägbara migrationsregler som stärker Sveriges position i det globala forskarsamhället och tryggar unga forskares framtid.


\section{Erfarenhet}

\textit{Beskriv 1-3 saker som du gjort i din akademiska karriär utöver forskningen som är relevant för Sveriges unga akademis verksamhet. (1000 tecken)}

1.  Samverkan och kommunikation

Jag arbetar aktivt för att göra forskning tillgänglig för allmänheten,
t.ex. genom en distanskurs om klimatet för allmänheten,
medverkan i media (bl.a. i Sveriges Radio och populärvetenskapliga artiklar)
samt framträdanden i evenemang som Vetenskapsfestivalen och Stora Hållbarhetsdagen.

2. Skrivarklubb

Jag grundade och har i 1,5 år lett en skrivarklubb för unga forskare inom avdelningen.
Genom bokdiskussioner och peer-review utvecklar deltagarna sitt akademiska språk och förmågan att bygga tydliga och starka forskningsnarrativ,
vilket har höjt kvaliteten på deras manuskript.

3. Styrelsemedlem i MERGE

Som styrelseledamot i det strategiska forskningsområdet MERGE bidrar jag till strategiska beslut om forskningsinriktning,
resursfördelning och samarbeten inom klimat- och jordsystemmodellering.
Genom detta arbete har jag fått värdefull erfarenhet av att arbeta med tvärvetenskapliga frågor och stärka sambandet mellan olika aktörer inom svensk forskning.

\end{document}
