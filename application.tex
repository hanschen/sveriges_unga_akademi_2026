\documentclass[11pt,a4paper]{article}

\usepackage{lmodern}
\usepackage{microtype}
\usepackage{parskip}

\usepackage[a4paper,margin=1in]{geometry}

\usepackage{sectsty}
\sectionfont{\large}

\begin{document}

\begin{center}
    \Large\bfseries
    Ansökan -- Hans Chen
\end{center}

\section{Motivering}

\textit{Motivera kort varför du vill bli ledamot i Sveriges unga akademi. (1000 tecken)}

Jag vill bli ledamot i Sveriges unga akademi för att bidra till att stärka vetenskapens roll i en tid då kritiskt tänkande utmanas.
Som klimatforskare har jag upplevt hur vetenskap kan polariseras och angripas.
Jag disputerade i USA,
ett land med stark forskningstradition,
och det har bekymrat mig djupt att bevittna de senaste årens systematiska angrepp på vetenskap och akademisk frihet där.
Detta visar hur sårbar vetenskapen är,
och hur avgörande det är att bygga tillit till forskningsprocessen i samhället.

Jag brinner även för att förbättra villkoren för unga forskare.
Akademiska karriärvägar måste bli mer transparenta och rättvisa.
Mina erfarenheter från USA, Sverige och Kina ger mig ett brett perspektiv på olika forskningspolitiska system och möjligheter.

Genom SUA vill jag bidra till tvärvetenskapliga initiativ som stärker förtroendet för forskning
och främjar kritiskt tänkande i samhället.
I en tid av desinformation är detta avgörande för demokrati och en hållbar framtid.



\section{Aktiviteter/Frågor - bidrag}

\textit{Ge exempel på aktiviteter och frågor där du tror att du kan bidra om du blir ledamot i Sveriges unga akademi. (1000 tecken)}

Som ledamot vill jag utveckla strategier för att stärka vetenskapens roll och motverka desinformation.
I dagens digitala samhälle bombarderas vi ständigt av information,
vilket gör det lätt att ta vetenskapliga framsteg som vaccin och väderprognoser för givna.
För att motverka detta vill jag initiera "vetenskapliga boosters"
– återkommande insatser i form av t.ex. korta videoklipp,
publika event
eller skolbesök,
som påminner om den vetenskapliga metodens och grundforskningens fundamentala värde.

Vidare vill jag stärka karriärstödet för unga forskare.
Tiden efter disputation är ofta fylld av osäkerhet.
Jag vill därför bidra till en guide riktad specifikt till nydisputerade och postdoktorer,
som ett fokuserat komplement till SUA:s "A Beginner's Guide to Swedish Academia".
Denna guide skulle kunna beskriva karriärvägar både inom och utanför akademin,
ge konkreta råd om ansökningar och strategisk kompetensutveckling,
samt dela erfarenheter från forskare i olika sektorer.


\section{Aktiviteter/Frågor – driva}

\textit{Ge exempel på aktiviteter och frågor du skulle vilja driva om du blir ledamot i Sveriges unga akademi. (1000 tecken)}

Som ledamot vill jag fokusera på två kritiska frågor för Sveriges framtid som forskningsnation.
För det första ser jag ett akut behov av att hjälpa allmänheten navigera den rådande AI-revolutionen.
Jag skulle vilja initiera arbetet med en lättillgänglig guide om AI för allmänheten som förklarar grundläggande koncept, möjligheter och risker,
samt hur man förhåller sig kritiskt till AI-genererad information.
Guiden kan riktas till olika målgrupper
– lärare, elever i grundskolan, föräldrar, beslutsfattare
– kompletterat med skolmaterial och publika seminarier.

Min andra prioritet är att stärka Sveriges förmåga att attrahera och behålla internationella toppforskare.
Många unga forskare möter idag byråkratiska hinder kring migrationsregler och social trygghet.
Ändringarna i utlänningslagen 2021 skapade exempelvis stor osäkerhet för internationella doktorander.
Jag vill att SUA aktivt driver dessa frågor genom policyrapporter, dialoger med myndigheter och lärosäten, samt regelförslag.


\section{Erfarenhet}

\textit{Beskriv 1-3 saker som du gjort i din akademiska karriär utöver forskningen som är relevant för Sveriges unga akademis verksamhet. (1000 tecken)}

1.  Samverkan och kommunikation

Jag arbetar aktivt för att göra forskning tillgänglig för allmänheten,
t.ex. genom en distanskurs om klimatet för allmänheten,
medverkan i media (bl.a. i Sveriges Radio och populärvetenskapliga artiklar)
samt framträdanden i evenemang som Vetenskapsfestivalen och Stora Hållbarhetsdagen.

2. Skrivarklubb

Jag grundade och har i 1,5 år lett en skrivarklubb för unga forskare inom avdelningen.
Genom bokdiskussioner och peer-review utvecklar deltagarna sitt akademiska språk och förmågan att bygga tydliga och starka forskningsnarrativ,
vilket har höjt kvaliteten på deras manuskript.

3. Styrelsemedlem i MERGE

Som styrelseledamot i det strategiska forskningsområdet MERGE bidrar jag till strategiska beslut om forskningsinriktning,
resursfördelning och samarbeten inom klimat- och jordsystemmodellering.
Genom detta arbete har jag fått värdefull erfarenhet av att arbeta med tvärvetenskapliga frågor och stärka sambandet mellan olika aktörer inom svensk forskning.

\end{document}
