\documentclass[10pt,a4paper]{article}

\usepackage{lmodern}
\usepackage{microtype}
\usepackage{enumitem}

\usepackage[a4paper,margin=1in]{geometry}


\usepackage{hyperref}
\urlstyle{same}
\hypersetup{
    hidelinks,
    colorlinks=false,
}

\usepackage{sectsty}
\sectionfont{\large}


\begin{document}

\begin{center}
    \Large\bfseries
    CV -- Hans Chen
\end{center}

\section{Högskoleexamen}

\begin{itemize}[itemsep=0pt]
    \item 2010: Kandidatexamen i meteorologi, Stockholms universitet.
    \item 2012: Masterexamen i meteorologi, oceanografi och klimatfysik, Stockholms universitet.
\end{itemize}


\section{Doktorsexamen}

\begin{itemize}[itemsep=0pt]
    \item 2018: Meteorologi och atmosfärsvetenskap, The Pennsylvania State University.

        Avhandlingens titel: \textit{Toward improved regional estimates of carbon dioxide sources and sinks through coupled carbon–atmospheric data assimilation}.

        Handledare: Fuqing Zhang.
\end{itemize}



\section{Postdoktorsvistelser}

\begin{itemize}[itemsep=0pt]
    \item 2018--2020: Lunds universitet.
\end{itemize}


\section{Docentkompetens}

\section{Nuvarande anställning}

\begin{itemize}[itemsep=0pt]
    \item Biträdande universitetslektor, Chalmers tekniska universitet.

        Tidsbegränsad anställning till 31 december 2026.
\end{itemize}


\section{Tidigare anställningar}

\begin{itemize}[itemsep=0pt]
    \item 2020--2022: Forskare, tillsvidareanställning.
\end{itemize}


\section{Uppehåll i forskningen}


\begin{itemize}[itemsep=0pt]
    \item Föräldraledighet: 46 dagar.
\end{itemize}


\section{Handledning}

\begin{itemize}[itemsep=0pt]
    \item 2025--idag: Doktorand, Fangyuan Cheng, \textbf{huvudhandledare}.
    \item 2023--idag: Doktorand, Erik Holmgren, \textbf{huvudhandledare}.
    \item 2025--idag: Postdoktor, Xinyu Yang, \textbf{huvudhandledare}.
    \item 2024--idag: Postdoktor, Hari Ram Chandrika Rajendran Nair, \textbf{huvudhandledare}.
    \item 2023--idag: Postdoktor, Ziqian Zhong, \textbf{huvudhandledare}.
    \item 2024--idag: Doktorand, Lloyd Villanueva, bi-handledare.
    \item 2023--2025, Doktorand, Hanna Hallborn, bi-handledare.
    \item 2020--2025, Doktorand, Ziyi Cai, bi-handledare.
\end{itemize}


\section{Erhållna forskningsmedel}

\begin{itemize}[itemsep=0pt]
    \item 2025--2030: Biomass Calibration/Validation, ESA, finansiering \textit{in natura} (fri tillgång till satellitdata), medsökande.
    \item 2024--2026: Unga forskare 2023, Hasselbladstiftelsen, belopp: 1\,000\,000 SEK, \textbf{huvudsökande}.
    \item 2024--2027: Karriärstöd för forskare tidigt i karriären, Formas, belopp: 4\,000\,000 SEK, medsökande.
        Finansierar en doktorand i projektet som jag är bi-handledare till.
    \item 2023--2026: Horizon Europe, European Commission, belopp: 5\,500\,000 EUR (min andel: 38\,469 EUR), task leader.
    \item 2022--2025: Call 2021-C, Rymdstyrelsen, belopp: 4\,200\,000 SEK, \textbf{huvudsökande}.
    \item 2022--2025: Karriärstöd för forskare tidigt i karriären, Formas, belopp: 4\,000\,000 SEK, \textbf{huvudsökande}
        (beviljat men avböjt på grund av överlappande karriärbidrag).
    \item 2021--2023: Projektfinansiering, Biodiversity and Ecosystem services in a Changing Climate, belopp: 1\,000\,000 SEK (min andel: 50\,000 SEK), medsökande.
    \item 2021--2024: Joint China-Sweden Mobility, STINT, belopp: 400\,000 SEK, \textbf{huvudsökande}.
\end{itemize}


\section{Övrig information av betydelse för ansökan}

\subsection*{Priser och utmärkelser}

\begin{itemize}[itemsep=0pt]
    \item 2020: NASA Group Achievement Award, som medlem i ACT-America team.
    \item 2018: Outstanding Student Paper Award, American Meteorological Society.
    \item 2014: Hans Neuberger Award, The Pennsylvania State University, pedagogiskt pris.
\end{itemize}


\subsection*{Förtroendeuppdrag}

\begin{itemize}[itemsep=0pt]
    \item 2024--2025: Styrelseledamot, strategiskt forskningsområde MERGE.
    \item 2025--idag: Editorial Board Member, npj Artificial Intelligence.
    \item 2024--idag: Associate Editor, Frontiers in Remote Sensing.
    \item 2022--idag: Young Editorial Member, Advances in Climate Change Research.
    \item 2021--2023: Deputy, Copernicus CO2 General Assembly.
\end{itemize}


\subsection*{Ledarskap}

\begin{itemize}[itemsep=0pt]
    \item 2023--idag: Gruppledare, Climate Dynamics Group (\url{https://climatedynamics.group}).
    \item 2024--2027: Huvudhandledare och vetenskaplig värd, Marie Skłodowska-Curie Actions Postdoctoral Fellowship (Ziqian Zhong).
\end{itemize}


\subsection*{Internationalisering}

\begin{itemize}[itemsep=0pt]
    \item 2025--2028: Finansiering för internationalisering,
        National Natural Science Foundation of China,
        medsökande.
    \item 2023--2026: Finansiering för internationalisering,
        Shanghai Action Plan for Science, Technology and Innovation, International Science and Technology Cooperation Program,
        medsökande.
    \item 2023 och 2025: Arrangör av workshop, ``Cycle in the Climate--Vulnerable Regions: Modeling and Observations''
        med tillhörande sommarskola i 2025, Kina.
\end{itemize}


\subsection*{Utåtriktad verksamhet}

\begin{itemize}[itemsep=0pt]
    \item 2025--idag: Återkommande gästföreläsare vid Hulebäcksgymnasiet.
    \item 2024: Öppningsanförande under Stora Hållbarhetsdagen Göteborg.
    \item 2023: Medverkan i paneldiskussion under Vetenskapsfestivalen.
    \item 2021--idag: Medverkan i media, bland annat Sveriges Radio och intervjuer i tidskrifter.
    \item 2013--idag: Vetenskaplig hemsida (\url{https://hanschen.org}), över 310\,000 besök totalt.
\end{itemize}

\subsection*{Samverkan med näringsliv eller kultursektorn}

\begin{itemize}[itemsep=0pt]
    \item 2025--idag: Vetenskaplig rådgivare för offentlig konst vid Korsvägen (Västlänken).

        Samarbete med Trafikverket och konstnären Henrik Håkansson för att utveckla ett konstverk relaterat till väder och klimat till den nya stationen.
    \item 2024--2025: Konsultuppdrag åt SPASCIA, om satelliter för koldioxidmätning, belopp: 5\,000 EUR.
\end{itemize}

\end{document}
